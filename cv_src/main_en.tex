% Based on overleaf template https://www.overleaf.com/latex/templates/rendercv-classic-theme/szbrrwnrfksk

\documentclass[10pt, letterpaper]{article}

% Packages:
\usepackage[
    ignoreheadfoot, % set margins without considering header and footer
    top=1 cm, % seperation between body and page edge from the top
    bottom=1 cm, % seperation between body and page edge from the bottom
    left=1.2 cm, % seperation between body and page edge from the left
    right=1.5 cm, % seperation between body and page edge from the right
    footskip=1.0 cm, % seperation between body and footer
    % showframe % for debugging 
]{geometry} % for adjusting page geometry
\usepackage[explicit]{titlesec} % for customizing section titles
\usepackage{tabularx} % for making tables with fixed width columns
\usepackage{array} % tabularx requires this
\usepackage[dvipsnames]{xcolor} % for coloring text
\definecolor{primaryColor}{RGB}{0, 79, 144} % define primary color
\usepackage{enumitem} % for customizing lists
\usepackage{fontawesome5} % for using icons
\usepackage{amsmath} % for math
\usepackage[
    pdftitle={Denis Bakin's CV},
    pdfauthor={Denis Bakin},
    pdfcreator={LaTeX with RenderCV},
    colorlinks=true,
    urlcolor=primaryColor
]{hyperref} % for links, metadata and bookmarks
\usepackage[pscoord]{eso-pic} % for floating text on the page
\usepackage{calc} % for calculating lengths
\usepackage{bookmark} % for bookmarks
\usepackage{lastpage} % for getting the total number of pages
\usepackage{changepage} % for one column entries (adjustwidth environment)
\usepackage{paracol} % for two and three column entries
\usepackage{ifthen} % for conditional statements
\usepackage{needspace} % for avoiding page brake right after the section title
\usepackage{iftex} % check if engine is pdflatex, xetex or luatex

% Ensure that generate pdf is machine readable/ATS parsable:
\ifPDFTeX
    \input{glyphtounicode}
    \pdfgentounicode=1
    \usepackage[T1]{fontenc}
    \usepackage[utf8]{inputenc}
    \usepackage{lmodern}
\fi

\usepackage[default, type1]{sourcesanspro} 

% Some settings:
\AtBeginEnvironment{adjustwidth}{\partopsep0pt} % remove space before adjustwidth environment
\pagestyle{empty} % no header or footer
\setcounter{secnumdepth}{0} % no section numbering
\setlength{\parindent}{0pt} % no indentation
\setlength{\topskip}{0pt} % no top skip
\setlength{\columnsep}{0.15cm} % set column seperation
\makeatletter
\let\ps@customFooterStyle\ps@plain % Copy the plain style to customFooterStyle
\patchcmd{\ps@customFooterStyle}{\thepage}{
    \color{gray}\textit{\small Denis Bakin - Page \thepage{} of \pageref*{LastPage}}
}{}{} % replace number by desired string
\makeatother
\pagestyle{customFooterStyle}

\titleformat{\section}{
    % avoid page braking right after the section title
    \needspace{4\baselineskip}
    % make the font size of the section title large and color it with the primary color
    \Large\color{primaryColor}
}{
}{
}{
    % print bold title, give 0.15 cm space and draw a line of 0.8 pt thickness
    % from the end of the title to the end of the body
    \textbf{#1}\hspace{0.15cm}\titlerule[0.8pt]\hspace{-0.1cm}
}[] % section title formatting

\titlespacing{\section}{
    % left space:
    -1pt
}{
    % top space:
    0.3 cm
}{
    % bottom space:
    0.2 cm
} % section title spacing

% \renewcommand\labelitemi{$\vcenter{\hbox{\small$\bullet$}}$} % custom bullet points
\newenvironment{highlights}{
    \begin{itemize}[
        topsep=0.10 cm,
        parsep=0.10 cm,
        partopsep=0pt,
        itemsep=0pt,
        leftmargin=0.4 cm + 10pt
    ]
}{
    \end{itemize}
} % new environment for highlights

\newenvironment{highlightsforbulletentries}{
    \begin{itemize}[
        topsep=0.10 cm,
        parsep=0.10 cm,
        partopsep=0pt,
        itemsep=0pt,
        leftmargin=10pt
    ]
}{
    \end{itemize}
} % new environment for highlights for bullet entries


\newenvironment{onecolentry}{
    \begin{adjustwidth}{
        0.2 cm + 0.00001 cm
    }{
        0.2 cm + 0.00001 cm
    }
}{
    \end{adjustwidth}
} % new environment for one column entries

\newenvironment{twocolentry}[2][]{
    \onecolentry
    \def\secondColumn{#2}
    \setcolumnwidth{\fill, 4.5 cm}
    \begin{paracol}{2}
}{
    \switchcolumn \raggedleft \secondColumn
    \end{paracol}
    \endonecolentry
} % new environment for two column entries

\newenvironment{threecolentry}[3][]{
    \onecolentry
    \def\thirdColumn{#3}
    \setcolumnwidth{1 cm, \fill, 4.5 cm}
    \begin{paracol}{3}
    {\raggedright #2} \switchcolumn
}{
    \switchcolumn \raggedleft \thirdColumn
    \end{paracol}
    \endonecolentry
} % new environment for three column entries

\newenvironment{header}{
    \setlength{\topsep}{0pt}\par\kern\topsep\centering\color{primaryColor}\linespread{1.5}
}{
    \par\kern\topsep
} % new environment for the header

\newcommand{\placelastupdatedtext}{% \placetextbox{<horizontal pos>}{<vertical pos>}{<stuff>}
  \AddToShipoutPictureFG*{% Add <stuff> to current page foreground
    \put(
        \LenToUnit{\paperwidth-2 cm-0.2 cm+0.05cm},
        \LenToUnit{\paperheight-1.0 cm}
    ){\vtop{{\null}\makebox[0pt][c]{
        \small\color{gray}\textit{Last updated: \today}\hspace{\widthof{Last updated: \today}}
    }}}%
  }%
}%

% save the original href command in a new command:
\let\hrefWithoutArrow\href

% new command for external links:
\renewcommand{\href}[2]{\hrefWithoutArrow{#1}{\ifthenelse{\equal{#2}{}}{ }{#2 }\raisebox{.15ex}{\footnotesize \faExternalLink*}}}


\begin{document}
    \newcommand{\AND}{\unskip
        \cleaders\copy\ANDbox\hskip\wd\ANDbox
        \ignorespaces
    }
    \newsavebox\ANDbox
    \sbox\ANDbox{}

    \placelastupdatedtext
    \begin{header}
        \fontsize{30 pt}{30 pt}
        \textbf{Denis Bakin}

        \vspace{0.3 cm}

        \normalsize
        \mbox{{\footnotesize\faMapMarker*}\hspace*{0.13cm}Moscow, Russia}%
        \kern 0.25 cm%
        \AND%
        \kern 0.25 cm%
        \mbox{\hrefWithoutArrow{https://t.me/dfbakin}{{\footnotesize\faTelegramPlane}\hspace*{0.13cm}dfbakin}}%
        \kern 0.25 cm%
        \AND%
        \kern 0.25 cm%
        \mbox{\hrefWithoutArrow{mailto:denis.bakin5@gmail.com}{{\footnotesize\faEnvelope[regular]}\hspace*{0.13cm}denis.bakin5@gmail.com}}%
        \kern 0.25 cm%
        % \AND%
        % \kern 0.25 cm%
        % \mbox{\hrefWithoutArrow{tel:+7-965-324-02-00}{{\footnotesize\faPhone*}\hspace*{0.13cm}+7-965-324-02-00}}%
        % \kern 0.25 cm%
        % \AND%
        % \kern 0.25 cm%
        % \mbox{\hrefWithoutArrow{https://yourwebsite.com/}{{\footnotesize\faLink}\hspace*{0.13cm}yourwebsite.com}}%
        % \kern 0.25 cm%
        % \AND%
        % \kern 0.25 cm%
        % \mbox{\hrefWithoutArrow{https://linkedin.com/in/yourusername}{{\footnotesize\faLinkedinIn}\hspace*{0.13cm}yourusername}}%
        % \kern 0.25 cm%
        \AND%
        \kern 0.25 cm%
        \mbox{\hrefWithoutArrow{https://github.com/dfbakin}{{\footnotesize\faGithub}\hspace*{0.13cm}dfbakin}}%
        \kern 0.25 cm%
    \end{header}

    \vspace{0.3 cm - 0.3 cm}


    \section{About me}
        
        \begin{onecolentry}
            I am searching for an internship in computer vision and machine learning. 
            As a motivated and detail-oriented student with a strong background in applied mathematics and programming, 
            I love working on real-time computer vision projects and developing software solutions. 
            I am always eager to learn, grow, and contribute to innovative projects in robotics and AI.
        \end{onecolentry}

    
    \section{Skills}



    \begin{onecolentry}
        \textbf{Technologies:} C++, Python, PyTorch, OpenCV, ROS2, Docker, Git, GitHub Actions
    \end{onecolentry}

    \vspace{0.2 cm}

    \begin{onecolentry}
        \textbf{Languages:} Russian (native), English (C1)
    \end{onecolentry}


    

    \section{Education}

        
        \begin{twocolentry}{
            Sept 2022 – Present
        }
            \textbf{Higher School of Economic}, Aplied Mathematics and Informatics
            \begin{highlights}
                \item Specialization: Machine Learning and Applications
            \end{highlights}
        \end{twocolentry}


    

    \section{Professional experience}


        \begin{twocolentry}{
        July 2023 – Present
        }
            \textbf{HSE Robotics Laboratory}, \href{https://cs.hse.ru/robotics/persons1}{Student}
            \begin{highlights}
                \item Worked on computer vision project "Handy" to detect and track table tennis ball in real-time
                \item Implemented a CI pipeline with GitHub Actions for the Lab's projects
            \end{highlights}
        \end{twocolentry}


        

    \section{Projects}

        
        \begin{twocolentry}{
            \href{https://github.com/robotics-laboratory/handy}{GitHub}
        }
            \textbf{Handy}
            \begin{highlights}
                \item Team project for HSE Robotics Laboratory to detect and track table tennis ball in real-time
                \item Tools Used: Python, OpenCV, PyTorch, ROS2
                \item Implemented a pipeline to read, syncronize and publish images from high-speed cameras
                \item Developed a calibration pipeline to obtain intrinsic parameters and perform stereo calibration (pinhole camera model was used)
                \item Developed a visualization tools for the project (\href{https://youtu.be/SzNMDVB9BTk}{official} and 
                 \href{https://drive.google.com/file/d/1aOPqKCeQD-07kMz72acdyqgvxf6h6TZo/view?usp=drive_link}{full} demo)
            \end{highlights}
        \end{twocolentry}


        \vspace{0.2 cm}

        \begin{twocolentry}{
            \href{https://github.com/dfbakin/TouchAndGo}{GitHub}
        }
            \textbf{TouchAndGo}
            \begin{highlights}  
                \item High school graduation project aimed at helping visually impaired individuals navigate public spaces  
                \item Developed a calibration pipeline for both internal and external calibration of a stereo camera pair  
                \item Generated depth maps from the stereo pair to simulate basic 2D "Lidar" data, which was then used to control vibrating motors on a wearable belt  
                \item Tools Used: Python, OpenCV, Raspberry Pi, and minor hardware modifications  
            \end{highlights}  
            
        \end{twocolentry}
   


    \section{Teaching}
        \begin{twocolentry}{
            Sept 2024 – Present
        }
            \textbf{HSE Lyceum}, Computer Science Teacher
                \begin{highlights}
                    \item Designed and developed a variety of programming problems
                    \item Created comprehensive teaching materials and resources to support student learning
                    \item Prepared and conducted exams to assess student understanding and progress
                    \item Assisted in developing programming problems for the enrollment examination process
                \end{highlights}
        \end{twocolentry}

        
        \begin{twocolentry}{
            Sept 2022 – Present
        }
            \textbf{Private Practice}, Tutor
            \begin{highlights}
                \item Mathematics, programming and computer science tutor for high school students
                \item State exam preparation for high school students
            \end{highlights}
        \end{twocolentry}
        \vspace{0.2 cm}  

\end{document}