% Based on overleaf template https://www.overleaf.com/latex/templates/rendercv-classic-theme/szbrrwnrfksk

\documentclass[a4paper,10pt]{article} 
\usepackage[T2A]{fontenc}
\usepackage[utf8]{inputenc}
\usepackage[russian]{babel}


\usepackage[
    ignoreheadfoot, % set margins without considering header and footer
    top=1 cm, % seperation between body and page edge from the top
    bottom=1 cm, % seperation between body and page edge from the bottom
    left=1.2 cm, % seperation between body and page edge from the left
    right=1.5 cm, % seperation between body and page edge from the right
    footskip=1.0 cm, % seperation between body and footer
    % showframe % for debugging 
]{geometry} % for adjusting page geometry
\usepackage[explicit]{titlesec} % for customizing section titles
\usepackage{tabularx} % for making tables with fixed width columns
\usepackage{array} % tabularx requires this
\usepackage[dvipsnames]{xcolor} % for coloring text
\definecolor{primaryColor}{RGB}{0, 79, 144} % define primary color
\usepackage{enumitem} % for customizing lists
\usepackage{fontawesome5} % for using icons
\usepackage{amsmath} % for math
\usepackage[
    pdftitle={Denis Bakin's CV},
    pdfauthor={Denis Bakin},
    pdfcreator={LaTeX with RenderCV},
    colorlinks=true,
    urlcolor=primaryColor
]{hyperref} % for links, metadata and bookmarks
\usepackage[pscoord]{eso-pic} % for floating text on the page
\usepackage{calc} % for calculating lengths
\usepackage{bookmark} % for bookmarks
\usepackage{lastpage} % for getting the total number of pages
\usepackage{changepage} % for one column entries (adjustwidth environment)
\usepackage{paracol} % for two and three column entries
\usepackage{ifthen} % for conditional statements
\usepackage{needspace} % for avoiding page brake right after the section title
\usepackage{iftex} % check if engine is pdflatex, xetex or luatex


% Ensure that generate pdf is machine readable/ATS parsable:
\ifPDFTeX
    \input{glyphtounicode}
    \pdfgentounicode=1
    \usepackage[T1]{fontenc}
    \usepackage[utf8]{inputenc}
    \usepackage{lmodern}
\fi

\usepackage{tempora} % Use Tempora font for better Cyrillic support

% Some settings:
\AtBeginEnvironment{adjustwidth}{\partopsep0pt} % remove space before adjustwidth environment
\pagestyle{empty} % no header or footer
\setcounter{secnumdepth}{0} % no section numbering
\setlength{\parindent}{0pt} % no indentation
\setlength{\topskip}{0pt} % no top skip
\setlength{\columnsep}{0.15cm} % set column seperation
\makeatletter
\let\ps@customFooterStyle\ps@plain % Copy the plain style to customFooterStyle
\patchcmd{\ps@customFooterStyle}{\thepage}{
    \color{gray}\textit{\small Denis Bakin - Page \thepage{} of \pageref*{LastPage}}
}{}{} % replace number by desired string
\makeatother
\pagestyle{customFooterStyle}

\titleformat{\section}{
    % avoid page braking right after the section title
    \needspace{4\baselineskip}
    % make the font size of the section title large and color it with the primary color
    \Large\color{primaryColor}
}{
}{
}{
    % print bold title, give 0.15 cm space and draw a line of 0.8 pt thickness
    % from the end of the title to the end of the body
    \textbf{#1}\hspace{0.15cm}\titlerule[0.8pt]\hspace{-0.1cm}
}[] % section title formatting

\titlespacing{\section}{
    % left space:
    -1pt
}{
    % top space:
    0.3 cm
}{
    % bottom space:
    0.2 cm
} % section title spacing

% \renewcommand\labelitemi{$\vcenter{\hbox{\small$\bullet$}}$} % custom bullet points
\newenvironment{highlights}{
    \begin{itemize}[
        topsep=0.10 cm,
        parsep=0.10 cm,
        partopsep=0pt,
        itemsep=0pt,
        leftmargin=0.4 cm + 10pt
    ]
}{
    \end{itemize}
} % new environment for highlights

\newenvironment{highlightsforbulletentries}{
    \begin{itemize}[
        topsep=0.10 cm,
        parsep=0.10 cm,
        partopsep=0pt,
        itemsep=0pt,
        leftmargin=10pt
    ]
}{
    \end{itemize}
} % new environment for highlights for bullet entries


\newenvironment{onecolentry}{
    \begin{adjustwidth}{
        0.2 cm + 0.00001 cm
    }{
        0.2 cm + 0.00001 cm
    }
}{
    \end{adjustwidth}
} % new environment for one column entries

\newenvironment{twocolentry}[2][]{
    \onecolentry
    \def\secondColumn{#2}
    \setcolumnwidth{\fill, 4.5 cm}
    \begin{paracol}{2}
}{
    \switchcolumn \raggedleft \secondColumn
    \end{paracol}
    \endonecolentry
} % new environment for two column entries

\newenvironment{threecolentry}[3][]{
    \onecolentry
    \def\thirdColumn{#3}
    \setcolumnwidth{1 cm, \fill, 4.5 cm}
    \begin{paracol}{3}
    {\raggedright #2} \switchcolumn
}{
    \switchcolumn \raggedleft \thirdColumn
    \end{paracol}
    \endonecolentry
} % new environment for three column entries

\newenvironment{header}{
    \setlength{\topsep}{0pt}\par\kern\topsep\centering\color{primaryColor}\linespread{1.5}
}{
    \par\kern\topsep
} % new environment for the header

\newcommand{\placelastupdatedtext}{% \placetextbox{<horizontal pos>}{<vertical pos>}{<stuff>}
  \AddToShipoutPictureFG*{% Add <stuff> to current page foreground
    \put(
        \LenToUnit{\paperwidth-2 cm-0.2 cm+0.05cm},
        \LenToUnit{\paperheight-1.0 cm}
    ){\vtop{{\null}\makebox[0pt][c]{
        \small\color{gray}\textit{Last updated: \today}\hspace{\widthof{Last updated: \today}}
    }}}%
  }%
}%

% save the original href command in a new command:
\let\hrefWithoutArrow\href

% new command for external links:
\renewcommand{\href}[2]{\hrefWithoutArrow{#1}{\ifthenelse{\equal{#2}{}}{ }{#2 }\raisebox{.15ex}{\footnotesize \faExternalLink*}}}

\begin{document}

\newcommand{\AND}{\unskip
\cleaders\copy\ANDbox\hskip\wd\ANDbox
\ignorespaces
}
\newsavebox\ANDbox
\sbox\ANDbox{}

\placelastupdatedtext

\begin{header}
    \fontsize{30 pt}{30 pt}
    \textbf{Денис Бaкин}

    \vspace{0.3 cm}

    \normalsize
    \mbox{{\footnotesize\faMapMarker*}\hspace*{0.13cm}Москва, Россия}%
    \kern 0.25 cm%
    \AND%
    \kern 0.25 cm%
    \mbox{\hrefWithoutArrow{https://t.me/dfbakin}{{\footnotesize\faTelegramPlane}\hspace*{0.13cm}dfbakin}}%
    \kern 0.25 cm%
    \AND%
    \kern 0.25 cm%
    \mbox{\hrefWithoutArrow{mailto:denis.bakin5@gmail.com}{{\footnotesize\faEnvelope[regular]}\hspace*{0.13cm}denis.bakin5@gmail.com}}%
    \kern 0.25 cm%
    \AND%
    \kern 0.25 cm%
    \mbox{\hrefWithoutArrow{https://github.com/dfbakin}{{\footnotesize\faGithub}\hspace*{0.13cm}dfbakin}}%
    \kern 0.25 cm%
\end{header}

\section{Обо мне}
    \begin{onecolentry}
        Я ищу стажировку в области компьютерного зрения и машинного обучения.
        Я мотивированный студент с опытом работы над проектами компьютерного зрения. Готов много учиться, развиваться и вносить вклад в команду.
    \end{onecolentry}

\section{Навыки}
    \begin{onecolentry}
        \textbf{Технологии:} C++, Python, PyTorch, OpenCV, ROS2, Docker, Git, GitHub Actions
    \end{onecolentry}
    \vspace{0.2 cm}
    \begin{onecolentry}
        \textbf{Языки:} Русский (носитель), Английский (C1)
    \end{onecolentry}

\section{Образование}
    \begin{twocolentry}{Сентябрь 2022 – н. в.}
        \textbf{Высшая школа экономики}, Прикладная математика и информатика
        \begin{highlights}
            \item Специализация: Машинное обучение и его приложения
        \end{highlights}
    \end{twocolentry}

\section{Опыт работы}
    \begin{twocolentry}{Июль 2023 – н. в.}
        \textbf{Лаборатория робототехники ВШЭ}, \href{https://cs.hse.ru/robotics/persons1}{Студент}
        \begin{highlights}
            \item Работал над проектом компьютерного зрения "Handy" для отслеживания движения теннисного мяча в реальном времени
            \item Реализовал CI-пайплайн с использованием GitHub Actions для проектов лаборатории
        \end{highlights}
    \end{twocolentry}

\section{Проекты}
    \begin{twocolentry}{\href{https://github.com/robotics-laboratory/handy}{GitHub}}
        \textbf{Handy}
        \begin{highlights}
            \item Командный проект в Лаборатории робототехники ВШЭ по отслеживанию теннисного мяча в реальном времени
            \item Технологии: Python, OpenCV, PyTorch, ROS2
            \item Разработал пайплайн для получения и синхронизации изображений с высокочастотныхтных камер и
                работы с многопоточностью в ROS2 (и затем без него)
            \item Создал калибровочный пайплайн для получения параметров камеры и стереокалибровки
            \item Разработал утилиту для визуализации процесса калибровки и результатов офлайн-детекции
        \end{highlights}
    \end{twocolentry}

    \vspace{0.2 cm}

    \begin{twocolentry}{\href{https://github.com/dfbakin/TouchAndGo}{GitHub}}
        \textbf{TouchAndGo}
        \begin{highlights}
            \item Выпускной проект школы, помогающий незрячим людям ориентироваться в общественных местах
            \item Разработал калибровочный пайплайн для стереопары камер
            \item Сделал модуль для генерации карты глубины для имитации 2D "лидарных" данных и управления вибромоторами на поясе
            \item Технологии: Python, OpenCV, Raspberry Pi, работа с аппаратным обеспечением
        \end{highlights}
    \end{twocolentry}

\section{Преподавательская деятельность}
    \begin{twocolentry}{Сентябрь 2024 – н. в.}
        \textbf{Лицей ВШЭ}, Преподаватель программирования
        \begin{highlights}
            \item Разработка задач по программированию
            \item Создание учебных материалов, проведение занятий
            \item Подготовка и проведение элементов контроля
            \item Участие в разработке задач для вступительных испытаний
        \end{highlights}
    \end{twocolentry}

    \begin{twocolentry}{Сентябрь 2022 – н. в.}
        \textbf{Частный репетитор}
        \begin{highlights}
            \item Преподавание математики, информатики и программирования
            \item Подготовка к экзаменам (в основном ЕГЭ)
        \end{highlights}
    \end{twocolentry}

\end{document}